\section{Evaluation}
\label{sec:evaluation}

In this section, we first give out some comparisons on accuracy. Then, to show the performance improvement of local feature descriptor by using {\sys}, an evaluation is also performed on {\sys} combined with SIFT and SURF, two most widely used {\lfea}s.

These two evaluations are performed in two different environments. For salient region comparison, we run algorithms on a PC with one Intel Quad Core 2.4Ghz CPU. For image retrieval system integration, the {\sys} is integrated to an exist image retrieval system on a server with Intel Octa-Core i7 3.4Ghz CPU. We also employ two benchmarks for our evaluation. The first one is a dataset of 1000 $640\times480$images with labeling salient regions by Achanta et at.~\cite{achanta2009frequency}, which is used to evaluate the salient region accuracy. The second one is a much larger image database provided by Nist~\cite{nister-stewenius-cvpr-2006}, which consists of 10200 $640\times480$ images for benchmarking image retrieval system.

\subsection{Experimental Comparison}
\label{sec:evaluation_comparison}

The major motivation of this paper is to design an algorithm for local feature reduction using salient region. To achieve this goal, an approximate salient region detection algorithm is designed. Although our approach cannot detect precise salient region, we are still interested in evaluating its precision against general purpose ones. Here we compare {\sys} with GBSR (Global Contrast based Salient Region)~\cite{cheng2011global}. GBSR is a state-of-the-art salient region detection algorithm, which can achieve the best precision and recall rate compared with the other salient region algorithms. Moreover, it is written in C++, which makes us easy to compare the computation efficiency with ours. The processing time of GBSR includes two parts: the saliency map detection and its segmentation.

Since we only focus on local features, our precision and recall evaluation differs from regular salient region researches. As input, we first generate both SURF and SIFT local features for the whole dataset. Then we collect the feature points in the salient regions calculated by the ground truth (labeling regions also by Achanta et al.), GBSR and {\sys}. At last, the GBSR's and {\sys}'s precision and recall are calculated based on their results compared with that of the ground truth. In other words, only the local features computed by the ground truth are considered as the truly salient features and the others are false features. Thus, the precision and recall equations are defined as follows.

{\begin{equation} \label{eq:precision}
Precision = \frac{{N}_{correct}}{{n}_{detected}}
\end{equation}}

{\begin{equation} \label{eq:recall}
Recall = \frac{{N}_{correct}}{{N}_{true}}
\end{equation}}

Where ${N}_{correct}$ refers to the number of local features located in both the ground truth and GBSR or {\sys}. ${N}_{detected}$ in equation~\ref{eq:precision} refers to the total number of local features detected by GBSR or {\sys}, while ${N}_{true}$ in equation~\ref{eq:recall} is the number of truly salient features in the ground truth.

\begin{table}[!ht]
\begin{center}
\begin{tabular}{|l|c|c|c|c|c|}
\hline
 & P & R & RE & SM(ms) & SG(ms) \\
\hline
GBSR   & 0.81 & 0.86 & 39\% &180 & 2080 \\
{\sys} & 0.52 & 0.59  & 42\% & N/A & 3 \\
\hline
\end{tabular}
\end{center}
\caption{Comparison between GBSR and {\sys} in precision (P), recall (R), reduction efficiency, saliency map (SM) and segmentation (SG) processing time in milliseconds.}
\label{tab:comparison}
\end{table}

The comparison result is presented in Table~\ref{tab:comparison}. As shown in the column 2, 3 and 4, {\sys} gets lower precision rate and lower recall rate compared to GBSR. The major reason is that our algorithm is an approximate design, which focuses on fast processing speed and accurate retrieval instead of precise boundaries of salient regions. The reduction efficiency in the column 5 refers to the proportion of the left local features after a reduction. The results shows that both of them can help to eliminate more than half of the original features. GBSR achieves a higher reduction rate due to its precise boundary detection.

The result in the column 6 of Table~\ref{tab:comparison} presents the efficiency comparison, It shows that the well optimized GBSR costs average 2086 milliseconds to get the final salient region for an image, while {\sys} only costs 3.2 milliseconds. Even only counting the time for saliency map detection, GBSR still needs average 178 milliseconds. The performance result of {\sys} is very impressive but not surprising, because the computation of {\sys} is really simple and straightforward.

\subsection{Combination into Retrieval Systems}
\label{sec:evaluation_integration}

\begin{table}[!ht]
\begin{center}
\begin{tabular}{|c|c|c|}
\hline
 & Query Time & Accuracy \\
\hline
SIFT & 1.5s & - \\
SIFT-{\sys} & 0.72s & 93\% \\
SIFT-GBSR & 2.58s & 85\% \\
\hline
SURF & 0.49s & - \\
SURF-{\sys} &  0.31s & 94\% \\
SURF-GBSR & 2.35s & 93\% \\
\hline
\end{tabular}
\end{center}
\caption{Comparison between original {\lfea}, {\sys} and GBSR conducted algorithms in a realistic image retrieval system. Query time includes salient region detection and feature detection time, VOC-Tree query time and RANSAC time for the whole image retrieval system. Accuracy refers to the accuracy of the {\sys} and GBSR conducted retrieval system compared to the original ones.}
\label{tab:integration}
\end{table}

\begin{figure*}[!ht]
\centering
\includegraphics[width=4.5in]{images/fig-gbsr.eps}
\caption{Example feature reduction results conducted by GBSR. From left to right, the first column lists original images, the second column presents binary masks detected by GBSR, and the third column are the local feature reduction result conducted by GBSR, where green points are salient features and red points are filtered ones.}
\label{fig:gbsr}
\end{figure*}

{\sys} is designed to improve the performance of {\lfea}s. To verify whether {\sys} satisfies this design purpose, we also evaluate the retrieval accuracy of {\sys} and GBSR. We implement a whole image retrieval system for evaluation. The system first builds a VOC-Tree~\cite{nister-stewenius-cvpr-2006} by using SIFT or SURF feature vectors extracted from image datasets. Then, each incoming query image is transferred into feature vectors too and retrieved in that VOC-Tree. At last, the found top similar images would be reordered by using RANSAC~\cite{ransac1981}. To involve {\sys}, we just replace the original local feature descriptor with a {\sys} enabled version. We also implement a GBSR integrated image retrieval system, which employs GBSR's precise salient region to conduct the local feature reduction.

Table~\ref{tab:integration} shows the final performance results. By reducing local features, the {\sys} conducted image retrieval system gains more than 2X speedup for one image query. Although GBSR version has a better reduction efficiency as discussed in the Section~\ref{sec:evaluation_comparison}, the query time becomes worse due to GBSR's long processing time.

We also present the accuracy influence by introducing {\sys} and GBSR to the retrieval system in the third column of Table~\ref{tab:integration}. Here, the retrieval results of different algorithms are scored following the method in \cite{nister-stewenius-cvpr-2006}. Then we get the accuracy by dividing the score of {\sys} and GBSR based algorithms by the score of the original ones. {\sys} based algorithms show a 7\% precision loss. In contrast, GBSR based algorithm shows a more obvious precision loss. When combined with SIFT, the performance loss is about 15\%. The major reason behind such a result is that GBSR is a precise silent region detection algorithm. When it used to reduce the feature points in a local feature-based retrieval system, only the points in its detection regions are remained. However, in {\lfea}s, many important points locate in the boundary of an object, which will have a great influence on the accuracy. To illustrate it clearer, we also give an example shown in Figure~\ref{fig:gbsr}. As the right column shown, many important feature points in the boundary are discarded after GBSR is used. 

