
%% lfsr.tex
%% 2012/07/24
%% by Max Lv

\documentclass[conference]{acm_proc_article-sp}


% *** CITATION PACKAGES ***
%
\usepackage{cite}
\usepackage[dvipdfm]{hyperref}


% *** GRAPHICS RELATED PACKAGES ***
%
\usepackage{graphicx}


% *** ALIGNMENT PACKAGES ***
%
\usepackage{array}


% *** SUBFIGURE PACKAGES ***
\usepackage[tight,footnotesize]{subfigure}
%\usepackage[caption=false,font=footnotesize]{subfig}


% *** PDF, URL AND HYPERLINK PACKAGES ***
%
\usepackage{url}


% correct bad hyphenation here
\hyphenation{op-tical net-works semi-conduc-tor}


%% *** DOCUMENT BEGINS HERE ***
%
\begin{document}


\title{Local Feature Based Salient Region Detection}
\author{
\alignauthor
Chao Lv\\
       \affaddr{Parallel Processing Institute\\Fudan University}\\
       \email{lch@fudan.edu.cn}
}
\maketitle


\begin{abstract}

Local feature descriptors have become the most important part in image / video retrieval systems. But considering the great amount of local features, which could be thousands of local features in one HD photo, it's hard to compute them efficiently in a realistic system. In our research, we try to overcome this obstacle with a straightforward local feature reduction processing by using salient region detection. Furthermore, we also present a efficient method to detect multi salient regions. The whole algorithm is only based on the local features, which means almost no additional overhead involved in our algorithm. In our tests, we compare the LFSR (Local Feature based Salient Region) algorithm with a state-of-the-art algorithm. And the LFSR shows a thousands of times speedup in runtime with acceptable precision and recall loss. When integrated with the SURF descriptor, LFSR can provide a overall 1.6X speedup for the whole computation.

\end{abstract}


\section{Introduction}

Local feature 

\hfill mds
 
\hfill January 11, 2007

\subsection{Subsection Heading Here}

Subsection text here.


\subsubsection{Subsubsection Heading Here}

Subsubsection text here.


% An example of a floating figure using the graphicx package.
% Note that \label must occur AFTER (or within) \caption.
% For figures, \caption should occur after the \includegraphics.
% Note that IEEEtran v1.7 and later has special internal code that
% is designed to preserve the operation of \label within \caption
% even when the captionsoff option is in effect. However, because
% of issues like this, it may be the safest practice to put all your
% \label just after \caption rather than within \caption{}.
%
% Reminder: the "draftcls" or "draftclsnofoot", not "draft", class
% option should be used if it is desired that the figures are to be
% displayed while in draft mode.
%
%\begin{figure}[!t]
%\centering
%\includegraphics[width=2.5in]{myfigure}
% where an .eps filename suffix will be assumed under latex, 
% and a .pdf suffix will be assumed for pdflatex; or what has been declared
% via \DeclareGraphicsExtensions.
%\caption{Simulation Results}
%\label{fig_sim}
%\end{figure}

% Note that IEEE typically puts floats only at the top, even when this
% results in a large percentage of a column being occupied by floats.


% An example of a double column floating figure using two subfigures.
% (The subfig.sty package must be loaded for this to work.)
% The subfigure \label commands are set within each subfloat command, the
% \label for the overall figure must come after \caption.
% \hfil must be used as a separator to get equal spacing.
% The subfigure.sty package works much the same way, except \subfigure is
% used instead of \subfloat.
%
\begin{figure*}[!t]
\centerline{\subfigure[Case I]{\includegraphics[width=2.5in,natwidth=400,natheight=300]{fig1.jpg}}
\hfil
\subfigure[Case II]{\includegraphics[width=2.5in,natwidth=400,natheight=300]{fig2.jpg}}}
\caption{Simulation results}
\end{figure*}
%
%Note that often IEEE papers with subfigures do not employ subfigure
%captions (using the optional argument to \subfloat), but instead will
%reference/describe all of them (a), (b), etc., within the main caption.


% An example of a floating table. Note that, for IEEE style tables, the 
% \caption command should come BEFORE the table. Table text will default to
% \footnotesize as IEEE normally uses this smaller font for tables.
% The \label must come after \caption as always.
%
\begin{center}
\begin{table*}[ht]
%% increase table row spacing, adjust to taste
\renewcommand{\arraystretch}{1.3}
% if using array.sty, it might be a good idea to tweak the value of
% \extrarowheight as needed to properly center the text within the cells
\caption{An Example of a Table}
\label{table_example}
%% Some packages, such as MDW tools, offer better commands for making tables
%% than the plain LaTeX2e tabular which is used here.
\centerline{\begin{tabular}{|c||c|}
\hline
AAAAAAAAAAAAAAAAAAAAAAAAAAAAAAAAAAA & BBBBBBBBBBBBBBBBBBBBBBBBBBBBBBBB\\
\hline
Three & Four\\
\hline
\end{tabular}}
\end{table*}
\end{center}


\section{Conclusion}
The conclusion goes here.


% use section* for acknowledgement
\section*{Acknowledgment}


The authors would like \cite{Gowers} to thank...


% trigger a \newpage just before the given reference
% number - used to balance the columns on the last page
% adjust value as needed - may need to be readjusted if
% the document is modified later
%\IEEEtriggeratref{8}


% references section
\bibliographystyle{abbrv}
\bibliography{lfsr}


% that's all folks
\end{document}


