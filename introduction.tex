\section{Introduction}
\label{sec:introduction}

Our society has entered a data-centric era with a huge amount of data being transferred and processed on the Internet. Among them, multimedia data, such as image and video, has become one of the major data type. As analyzed by CISCO Inc., video data occupies 50\% of network traffic in 2011 and will increase to 90\% in 2013~\cite{index2010forecast}.  According to a report~\cite{jansohn2009detecting}, as one of the most popular video sharing sites, more than 60-hour new videos are uploaded to \emph{YouTube} every minute. Moreover, \emph{Facebook} and \emph{Flickr} have hosted billions of user-uploaded photos respectively.

With the rapid increase of multimedia data, one of the most significant challenges is to understand and interpret such a huge amount of multimedia data. Currently, more and more retrieval applications are emerging to process these multimedia data, such as video recommendation~\cite{videorecommendation2007}, travel guidance~\cite{travelguidance2010} and content-based TV copy identification~\cite{tvidentify2003}. 

Based on feature types, these multimedia retrieval systems can be classified into two categories: global feature-based and local feature-based. Global feature-based algorithms tend to describe an image\footnote{Since video retrieval applications also use image retrieval algorithms to extract the features for their frames, these applications will be considered as special image retrieval applications in the following parts of this paper.} as a whole, such as contour representations, shape descriptors and texture features. Although global feature-based algorithms can achieve a high processing speed, their accuracy cannot be guaranteed. On the other hand, local feature-based algorithms represent  an image with hundreds of feature points, such as SIFT~\cite{lowe1999object,lowe2004distinctive} and SURF~\cite{Bay2006SURF,Evans20009OpenSURF}. 

Compared to global feature-based algorithms, local features are more robust, both scale-invariant and rotation-invariant~\cite{mikolajczyk2005performance}\cite{Bauer2007Evaluation}. However, even the SURF algorithm, an optimized algorithm derived from SIFT, is still very slow. While executed on a 3.3GHz Core i7 CPU~\cite{Fang2011ispass}, it can only achieve a processing speed of  about 2.6 frames per second , far from the real-time requirement. Moreover, since extracting hundreds of high-dimensional feature points for each image, it generally requires several KB storage space to save the feature points of an image. With a dramatically increasing of image or video amount on the Internet, it puts a great pressure on real-time processing and large-scale data storage.

In general, a local feature extraction algorithm consists of two stages: feature detection and feature description. In feature detection stage, feature points in an image are detected. And in description stage, each point is described into a high-dimensional vector based on the information around it. As analyzed in \cite{adaptivepipelineicpp2012}, description stage is more time-consuming. Therefore, less feature points means less processing time and less storage space. Salient region techniques, which picks up the visual attention parts from an image, can be used to reduced the amount of feature points through marking the features outside the region as unimportant and eliminating them. We have a comprehensive analysis on the relation between the salient region and the distribution of local features. In the analysis, we observe that local features in salient regions are more important than others outside the regions. Thus, in this paper, we present a Salient Region conducted Local Feature algorithm~({\sys}) which employs salient regions to eliminate local features that are not salient enough and extracts much fewer local features that need to be computed and stored.

However, general purpose salient region algorithms~\cite{cheng2011global,achanta2009frequency,itti1998model} do not meet our requirements. They have two major disadvantages for boosting local feature algorithms. First, these salient region algorithms are designed to provide precise region boundaries, which is not necessary and even harmful for local feature reduction: local features mostly locate on objects' edges and corners where are also boundaries of salient regions, which means many real salient regions could be eliminated incorrectly. Second, state-of-the-art salient region detection approaches are much slower than local feature algorithms, and it's impossible to integrate them together to improve the local feature algorithms' efficiency.

To overcome these obstacles, we involve an approximate algorithm instead of precise ones. This kind of salient region is also based on our observations: with the knowledge of the geometric distribution of local features, salient regions can be figured out in a very efficient way. This approach has two remarkable advantages: no additional image features for salient region detection; totally integrated with local feature algorithm and enable to compute the salient region efficiently.

The rest of the paper is organized as follows. Section~\ref{sec:observation} presents the base motivation and observation for our algorithm. Section~\ref{sec:algorithm} discusses the detailed algorithm. Several evaluations are presented in Section~\ref{sec:evaluation}. We conclude the paper in Section~\ref{sec:conclusion}.