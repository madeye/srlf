\section{Introduction}
\label{sec:introduction}

Our society has entered a data-centric era with a huge amount of data being transferred and processed on the Internet. Among them, multimedia data, such as image and video, has become one of the major data type. As analyzed by CISCO Inc., video data occupies 50\% of network traffic in 2011 and will increase to 90\% in 2013~\cite{index2010forecast}.  According to a report~\cite{jansohn2009detecting}, as one of the most popular video sharing sites, more than 60-hour new videos are uploaded to \emph{YouTube} every minute. Moreover, \emph{Facebook} and \emph{Flickr} have hosted billions of user-uploaded photos respectively.

With the rapid increase of multimedia data, one of the most significant challenges is to understand and interpret such a huge amount of multimedia data. Currently, more and more retrieval applications are emerging to process these multimedia data, such as video recommendation~\cite{videorecommendation2007}, travel guidance~\cite{travelguidance2010} and content-based TV copy identification~\cite{tvidentify2003}. In these systems, a fundamental step is to extract feature information from images. 

Image feature extraction algorithms includes two main domains : global feature-based and local feature-based. Global feature-based algorithms tend to describe an image as a whole, such as contour representations, shape descriptors and texture features. Although global feature-based algorithms can achieve a high processing speed, their accuracy cannot be guaranteed. On the other hand, local feature-based algorithms represent  an image with hundreds of feature points, such as SIFT~\cite{Lowe2004SIFT,RobHess} and SURF~\cite{Bay2006SURF,Evans20009OpenSURF}. 

Compared to global feature-based algorithms, local feature descriptors are more robust, both scale-invariant and rotation-invariant~\cite{mikolajczyk2005performance}\cite{Bauer2007Evaluation}. However, even the SURF descriptor, an optimized algorithm derived from SIFT, its processing speed is still slow. While executed on a 3.3GHz Core i7 CPU~\cite{Fang2011ispass}, it can only achieve a processing speed of  about 2.6 frames per second , far from real-time requirement. Moreover, since extracting hundreds of multi-dimensional feature points for each image, it generally requires several KB storage space to save the feature points of an image. With a dramatically increasing of image or video amount on the Internet, it puts a great pressure on real-time processing and large-scale data storage.

A local feature-based algorithm generally consists of two stages: feature detection and feature description. In feature detection stage, the points in  an image is detected. And in description stage, each point is described into a multi-dimensional vector based on the information around it. As analyzed in \cite{adaptivepipelineicpp2012}, description stage is more time-consuming. Therefore, less feature points means  less processing time and less storage space. Salient region techniques, which picks up the visual attention parts from an image, can be used to reduced the amount of feature points through marking the features outside the region as unimportant and eliminating them. Although there have existed many mature salient region algorithms~\cite{cheng2011global,achanta2009frequency,itti1998model}, they are not designed to work with local features. However, prior salient region detection algorithms are independent on the process of local feature-based algorithms. Therefore,  combining them would involve additional overhead.


To overcome these obstacles, we make a comprehensive analysis on the relation between the salient region and the distribution of feature points  in images. We observe that a salient region has more dense local features compared to other regions. Based on the observation, we design and implement a local feature-based salient region detection algorithms~({\sys}). After the distribution of local features are gotten, the regions with the most dense feature points are computed and chosen as the salient regions. And only the feature points in these salient regions will be described as the final feature points.  Experimental results show that {\sys} provides a thousands of times computation speedup with a similar accuracy compared to a state-of-art salient region algorithm. It achieves a processing speed of about 0.003s per image. Furthermore, when integrated with a local feature descriptor, {\sys} can achieve an overall 1.6X speedup with more than 50\% local features reduction.


%we present a algorithm named LFSR (Local Feature based Salient Region) to eliminate unimportant local features efficiently in this paper. 

%The main idea of LFSR algorithm is extracting salient regions of images and only describing the local features in those regions. Without involving a precise salient region algorithm, LFSR is only based on the extracted local features, no other image features needed. This approach has two remarkable advantages: no additional computation for salient region detection; totally integrated with local feature algorithm to compute the salient region efficiently. The details of LFSR can be found in Section~\ref{sec:algorithm}. 

%To prove our approximate algorithm is precise enough for local feature reduction, we evaluate LFSR against a stat-of-the-art salient region algorithm~\cite{achanta2009frequency}. The result shows that our approach has a much better performance with compatible precision and recall. Furthermore, within a realistic image retrieval system, our evaluation shows that LFSR can help to improve the computation performance significantly.

The rest of the paper is organized as follows. Section~\ref{sec:observation} presents the base motivation and observation for our algorithm. Section~\ref{sec:algorithm} discusses the detailed algorithm. Several evaluations are presented in Section~\ref{sec:evaluation}. We conclude the paper in Section~\ref{sec:conclusion}.