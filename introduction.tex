\section{Introduction}
\label{sec:introduction}

Our society has entered a data-centric era with a huge amount of data being transferred and processed on the Internet. Among them, multimedia data, such as image and video, has become one of the major data type. As analyzed by CISCO Inc., video data occupies 50\% of network traffic in 2011 and will increase to 90\% in 2013~\cite{index2010forecast}.  According to a report~\cite{jansohn2009detecting}, as one of the most popular video sharing sites, more than 20-hour new videos are uploaded to \emph{YouTube} every minute. Moreover, \emph{Facebook} and \emph{Flickr} have hosted billions of user-uploaded photos respectively.

With the rapid increase of multimedia data, one of the most significant challenges is to understand and interpret such a huge amount of multimedia data. Currently, more and more retrieval applications are emerging to process these multimedia data, such as video recommendation~\cite{videorecommendation2007}, travel guidance systems~\cite{travelguidance2010} and content-based TV copy identification~\cite{tvidentify2003}. In these systems, a fundamental step is to extract feature information from images. 

Image features includes two main domains -- global features and local features. Global image features tend to describe the image as a whole, such as contour representations, shape descriptors and texture features. The major limitation of global features is that they are not precise enough for some practical scenarios, for example, image retrieval systems. On the other hand, local image features represent image patches, computed at multiple points in the image. SIFT~\cite{Lowe2004SIFT,RobHess} and SURF~\cite{Bay2006SURF,Evans20009OpenSURF} are two most widely-used local image feature descriptors~\cite{Mikolajczyk2005Evaluation}\cite{Bauer2007Evaluation}. They extract a large collection of features as maxima and minima of the result of difference of Gaussians function applied in scale space to a series of smoothed and re-sampled images.

Compared to the global features, local feature descriptors are more robust, both scale-invariant and rotation-invariant~\cite{mikolajczyk2005performance}\cite{Bauer2007Evaluation}. But even the SURF descriptor, which is an optimized algorithm derived from SIFT, is still very slow in a practical usage -- the processing speed of SURF is about 2.6 frames per second on a 3.3GHz Core i7 CPU~\cite{Fang2011ispass}, far from the requirement of real time. Actually, for a typical local feature descriptor, local features can be more than thousands in a standard VGA ($640\times480$) image. Plus the computation of each local feature, the great amount of local features means a relatively high overhead. Furthermore, in a local feature based image retrieval system, the whole performance is mostly dominated by the quantity of features in the database, which can be more than billions of high-dimension vectors for a medium sized (millions) image dataset.

As a result, it's possible to improve the performance of the retrieval system by just reducing local features extracted from each image. A straightforward is detecting salient region then mark features outside that region as unimportant and eliminate them. Although there have existed many mature salient region algorithm, they are not designed to work with local features and have to introduce additional image feature computing. In this paper, we present a algorithm named LFSR (Local Feature based Salient Region) to eliminate unimportant local features efficiently.

The main idea of LFSR algorithm is extracting salient regions of images and only describing the local features in those regions. Without involving a precise salient region algorithm, LFSR is only based on the extracted local features, no other image features needed. This approach has two remarkable advantages: no additional computation for salient region detection; totally integrated with local feature algorithm to compute the salient region efficiently. The details of LFSR can be found in Section~\ref{sec:algorithm}. 

To prove our approximate algorithm is precise enough for local feature reduction, we evaluate LFSR against a stat-of-the-art salient region algorithm~\cite{achanta2009frequency}. The result shows that our approach has a much better performance with compatible precision and recall. Furthermore, within a realistic image retrieval system, our evaluation shows that LFSR can help to improve the computation performance significantly.

The rest of the paper is organized as follows. Section~\ref{sec:observation} presents the base motivation and observation for our algorithm. Section~\ref{sec:algorithm} discusses the detailed algorithm. Several evaluations are presented in Section~\ref{sec:evaluation}. We conclude the paper in Section~\ref{sec:conclusion}.