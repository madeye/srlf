\section{Introduction}
\label{sec:introduction}

Our society has entered a data-centric era and a huge amount of data are transferred and processed on the Internet. Among them, multimedia data, such as image and video, has become one of the major data types being processed. As analyzed by CISCO Inc., video data occupies 50\% of network traffic in 2011 and will increase to 90\% in 2013~\cite{CISCO2011}.  According to a report~\cite{youtube2009}, as one of the most popular video sharing sites, more than 20-hour new videos are uploaded to \emph{YouTube} every minute. Moreover, as two most popular photo sharing sites, \emph{Facebook} and \emph{Flickr} host billions of user-uploaded images respectively.

With the rapid increase of multimedia data, one of the most significant challenges is to understand and interpret such a huge amount of multimedia data. Currently, more and more retrieval applications are emerging to process these multimedia data, such as video recommendation~\cite{videorecommendation2007}, travel guidance systems~\cite{travelguidance2010} and content-based TV copy identification~\cite{tvidentify2003}. In these systems, a fundamental step is to extract feature information from images. 

Image features can be divided into two domains -- local features and global features. Global image features tend to describe the image as a whole, such as contour representations, shape descriptors and texture features. On the other hand, local image features represent image patches, computed at multiple points in the image. For example, SIFT~\cite{Lowe2004SIFT} and SURF~\cite{Bay2006SURF} are two most widely-used local image feature descriptors~\cite{Mikolajczyk2005Evaluation}\cite{Bauer2007Evaluation}. They use histograms of gradient orientations to extract feature points and describe them using high-dimension feature vectors.

Compared to the global features, local feature descriptors are more robust, both scale-invariant and rotation-invariant. But even the SURF descriptor, which is an optimized algorithm for SIFT, is still very slow in a practical usage -- the processing speed of SURF is about 2.6 frame per second on a 3.3GHz Core i7 CPU~\cite{Fang2011ispass}, far from the requirement of real time. Actually local feature descriptor should extract enough features from one image, and these features can be more than thousands for a standard VGA ($640\times480$) image. Considering the computation of describing each local feature, the great amount of local features means a relatively high overhead.

In general, there exist three major computation phases when processing local image features in a typical image retrieval system. First, the system detect all feature points from images. Then, with some specific formats and algorithms, each feature point is described as a high dimension vector. At last, all extracted feature are compared to each other according to the distance of their vectors.

According to a previous research~\cite{Fang2011ispass}, the computation of feature describing are obviously greater than the feature detecting in the SURF algorithm, which is caused by the great amount of local features to be described in one image. Furthermore, in a realistic image retrieval system, the performance is dominated by the number of features in the database.

So, it's possible and necessary to improve the performance of the whole system by reducing local features extracted from each image. In this paper, we present a algorithm named LFSR (Local Feature based Salient Region) to eliminate unimportant local features efficiently with no obvious precision loss for an image retrieval system.

The main idea of LFSR algorithm is extracting salient regions of images and only describing the local features in those regions. Without involving a precise salient region algorithm, LFSR is only based on the local features extracted from the first phase of local feature processing. This approach has two remarkable advantages: no additional computation for salient region detection; totally integrated with local feature algorithm to compute the salient region efficiently. The details of LFSR can be found in Section~\ref{sec:algorithm}. 

We have evaluated LFSR against a stat-of-the-art salient region algorithm~\cite{achanta2009frequency}. The evaluation results show that our approach has a much better performance with compatible precision and recall. Furthermore, withing a realistic image retrieval system, our evaluation shows that LFSR can help to improve both performance and accuracy.

The rest of the paper is organized as follows. Section~\ref{sec:observation} presents the base observations for our algorithm. Section~\ref{sec:algorithm} discusses the detailed algorithm. A evaluation is presented in Section~\ref{sec:evaluation}. We conclude the paper in Section~\ref{sec:conclusion}.