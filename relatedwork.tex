\section{Related Work}

There exists many kinds of local feature descriptors. SIFT~\cite{lowe1999object}\cite{lowe2004distinctive} is the most publicly accepted and robust local feature-based image extraction algorithm. For different conditions or purposes, some variants of SIFT are also widely used, such as RIFT~\cite{lazebnik2005sparse}, GLOH~\cite{mikolajczyk2005performance}, and PCA-SIFT~\cite{ke2004pca}. To deal with the performance issue of SIFT, SURF~\cite{Bay2006SURF} is proposed in 2006 as a optimized local feature descriptor based on SIFT. With acceptable precision loss, SURF tends to be an efficient alternative of SIFT. Most of these algorithms are insensitive to various transformations, such as scaling, rotation and illumination.

Most salient region researches focus on generating precise salient map. Some of them~\cite{cheng2011global,achanta2009frequency} have already achieved impressive precision and recall. Since salient region is really helpful to local feature descriptors, researchers have also tried to combine them together. Huang et al.~\cite{huang2009image} involves a salient region detection algorithm by Itti et al.~\cite{itti1998model} to eliminate nosies for SURF descriptor. Liang et al.~\cite{liang2010salient} uses similar salient map method to perform noise reduction for SIFT descriptor. But none of them try to reuse local features for salient region detection or consider improve the computation performance of local feature descriptors.